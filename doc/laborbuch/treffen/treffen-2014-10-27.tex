\chapter{Treffen am 27.10.2014}
\section{Anwesende Personen}
\begin{itemize}
	\item Stefan Biereigel
	\item Rene Winkler
	\item Mathias Fugmann
	\item Junlong Yin
\end{itemize}

\section{Besprochen bzw. Bearbeitet}
\begin{itemize}
	\item Design-Schritt 2
	\begin{itemize}
		\item Überarbeitung der Konzepte, rückblickende Diskussion
		\item siehe \autoref{fig:blau_v2}, \autoref{fig:gruen_v2}, \autoref{fig:rot_v2}.
		\item Feedback zum Laborbuch erhalten, rückblickende Überarbeitung, hinzufügen von Diskussionspunkten
	\end{itemize}
	\item Beginn der Bearbeitung von Design-Schritt 3
	\item Herausarbeiten von Themen zur näheren Behandlung im Seminar
	\begin{itemize}
		\item Beschleunigungssensoren (Ansteuerung, Auswertung)
		\item AD-Wandler im STM32F4
		\item SPI- und UART-Schnittstellen
		\item Interrupts im STM32F4
		\item LC-Displays (Ansteuerung, Datenformat, ...)
	\end{itemize}
\end{itemize}

\section{Entscheidungskriterien zur Beurteilung der Konzepte}
\begin{itemize}
	\item Preis
	\item Bedienbarkeit
	\item Komplexität
	\item Ähstetik
	\item Funktionsumfang
	\item Zuverlässigkeit
	\item Beschaffbarkeit der Teile
\end{itemize}