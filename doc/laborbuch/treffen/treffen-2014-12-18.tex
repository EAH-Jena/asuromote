\chapter{Treffen am 18.12.2014}
\section{Anwesende Personen}
\begin{itemize}
	\item Stefan Biereigel
	\item Rene Winkler
	\item Mathias Fugmann
	\item Junlong Yin
\end{itemize}

\section{Besprochen bzw. Bearbeitet}
Es wurde abschließend zur Entscheidungsmatrix diskutiert und wir kamen zusammen zu dem in \autoref{tab:entscheidung} dargestellten Ergebnis. Bewertet wurde nach dem Schulnotensystem, d.h. "1" ist die beste, "6" die schlechteste Bewertung. Das Konzept mit der niedrigsten Gesamtpunktzahl ist das beste.

\begin{table}[H]
\centering
\begin{tabular}{l|l|l|l}
 				& Blau 	& Grün 	& Rot	\\ \hline
Preis			& 6 	& 1		& 3		\\ \hline
Bedienbarkeit	& 1		& 4		& 2		\\ \hline
Komplexität		& 4		& 2		& 4		\\ \hline
Ähstetik		& 1		& 1		& 3		\\ \hline
Funktionsumfang	& 1		& 4		& 2		\\ \hline
Zuverlässigkeit	& 2		& 2		& 2		\\ \hline
Beschaffbarkeit	& 1		& 1		& 1		\\ \hline \hline
Gesamt			& 16	& 15	& 17	\\ \hline
\end{tabular}
\caption{Entscheidungsmatrix}
\label{tab:entscheidung}
\end{table}

Es zeigte sich schnell, dass die Konzepte alle ungefähr gleich gut abgeschnitten hatten. Jedes Konzept hat seine Vor- und Nachteile, die oft zwischen Preis und Funktionsumfang eintauschbar sind. Wir haben uns daher auf die Entwicklung eines neuen, gesamtheitlichen Konzepts geeinigt. 
Es vereint die preislichen Vorteile in der Übertragungstechnik durch Nutzung der nRF-Funkmodule und den Funktionsumfang (der hauptsächlich durch Software erweitert werden kann) durch Nutzung des Beschleunigungssensor. Es entspricht im wesentlichen dem grünen Konzept mit den eingezeichneten Erweiterungen.