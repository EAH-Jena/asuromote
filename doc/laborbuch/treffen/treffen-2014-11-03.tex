\chapter{Treffen am 3.11.2014}
\section{Anwesende Personen}
\begin{itemize}
	\item Stefan Biereigel
	\item Rene Winkler
	\item Mathias Fugmann
	\item Junlong Yin
\end{itemize}

\section{Besprochen bzw. Bearbeitet}
Weil es für das fertige Produkt mit großer Wahrscheinlichkeit benötigt wird, ein SPI-Interface am STM32F4 und den Beschleunigungssensor auf dem Discovery-Board anzusteuern, entschieden wir uns für die Entwicklung eines Testmoduls für ebendiese Aufgaben. Als Ziel soll ein kompaktes Modul mit dokumentierten Schnittstellen sein.

Verantwortlich für die Kodierung ist Rene Winkler, der Rest der Gruppe beschäftigt sich aber ebenfalls mit der notwendigen Theorie. Eine abschließende Präsentation der Ergebnisse vor den anderen Gruppenmitglieden wurde angedacht.

Der Rest des Praktikums wurde für die Beschäftigung mit den Datenblättern und Beginn der Kodierung verwendet.