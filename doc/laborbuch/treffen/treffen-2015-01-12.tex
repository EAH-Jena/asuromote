\chapter{Treffen am 12.01.2014}
\section{Anwesende Personen}
\begin{itemize}
	\item Stefan Biereigel
	\item Rene Winkler
	\item Mathias Fugmann
	\item Junlong Yin
\end{itemize}

\section{Besprochen bzw. Bearbeitet}
Stefan und Junlong besprachen das weitere Vorgehen für das abzugebende Arbeitspaket (Messung, Filterung, Auswertung Beschleunigungssensor). Folgendes Vorgehen wurde besprochen:
\begin{itemize}
	\item Inbetriebnahme bestehender Code Beschleunigungssensor - OK
	\item Inbetriebnahme bestehender Code UART (Mit USB-Seriell-Wandler)
	\begin{itemize}
		\item Probleme traten im Zusammenhang mit der Baudrate auf
		\item evtl. ist der Quarz auf dem STM32F4 defekt/out-of-spec?
		\item Versch. Baudraten brachten keine Besserung
		\item Problem: Einzelne Bits sind falsch, teilweise Offset von 0x40 vorhanden
	\end{itemize}
	\item Ausgabe Beschleunigungs-Sensor-Werte auf UART
	\item statistische Auswertung am PC mit Excel, Matlab, etc.
\end{itemize}

Rene und Mathias beschäftigten sich weiter mit der Designstudie für das Gehäuse und besprachen mit Herr Voß verschiedene, evtl. kritische Punkte.