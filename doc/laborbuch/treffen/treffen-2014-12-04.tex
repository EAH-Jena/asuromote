\chapter{Treffen am 04.12.2014}
\section{Anwesende Personen}
\begin{itemize}
	\item Stefan Biereigel
	\item Rene Winkler
	\item Mathias Fugmann
	\item Junlong Yin
\end{itemize}

\section{Besprochen bzw. Bearbeitet}
Stefan und Mathias besprachen offene Fragen zur elektrischen Verbindung des nRF-Funkmoduls mit dem STM32F4-Discovery-Board. Herr Voß verteilte ein weiteres Discovery-Board an die Gruppe, sodass nun auch mehr als eine Person zuhause Entwicklungen ausprobieren kann.
Stefan nahm die USB-UART-Kommunikation mit Hilfe der von ST bereitgestellten Library zur Verfügung. Diese funktioniert auf Linux auf Anhieb (Treiber für die USB-CDC-Klasse sind im Kernel integriert), führte aber auf Windows (7 und 8) anfangs zu Problemen. Ein korrekt funktionierender Treiber muss erst noch gefunden werden.
Es wurde ausgemacht, dass Stefan über das Wochenende eine Ansteuerung für den UART programmieren sollte.
Da Herr Voß anrat, sich weiter mit der eigentlichen Produktplanung zu beschäftigen, wird dies für das nächste Seminar eingeplant, um noch vor Weihnachten die Entscheidung für ein Konzept treffen zu können, welches dann umgesetzt werden kann. Auch wurden problematische Fragen beim Konzept der USB-Programmierung des Asuro untersucht. 