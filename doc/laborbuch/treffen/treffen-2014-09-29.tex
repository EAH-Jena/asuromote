\chapter{Treffen am 29.09.2014}
\section{Anwesende Personen}
\begin{itemize}
	\item Stefan Biereigel
	\item Rene Winkler
	\item Mathias Fugmann
	\item Junlong Yin
\end{itemize}

\section{Besprochen bzw. Bearbeitet}
\begin{itemize}
	\item Kundengespräch mit Herrn Voß
		\begin{itemize}
			\item Aufgabenstellung erhalten
			\item Kostenrahmen definiert
		\end{itemize}
	\item Erarbeitung der Spezifikation
\end{itemize}

\section{Problemstellung des Kunden}
\begin{itemize}
	\item Übertragung von Daten zum ASURO zu störanfällig (andere ASURO, Umwelteinflüsse)
	\begin{itemize}
		\item Diskussion: Kam uns nicht sehr störanfällig vor
		\item Fehlererkennung und -korrektur ist bereits vorhanden
		\item könnte man auch weiternutzen (über anderes Übertragungsmedium?)
	\end{itemize}
	\item Fernsteuerung für den ASURO fehlt
	\begin{itemize}
		\item erste Idee: WiiMote-ähnliche Steuerung wäre ansprechend
	\end{itemize}
\end{itemize}


\section{Spezifikation}
Legende: F = Forderung, W = Wunsch
\begin{itemize}
	\item Hauptanforderungen
	\begin{itemize}
		\item (F) Steuerung des ASURO über min. eine Eingabemöglichkeit
		\item (F) Programmierung des ASURO (Flashen) über eine Schnitstelle	
	\end{itemize}

\item Leistungsfähigkeit (Performance)
	\begin{itemize}
		\item (F) Steuerung aller Bewegungsrichtungen
		\item (F) Kommunikation mit ASURO fehlertolerant / nicht störanfällig
		\item (F) Kommunikationsreichweite minimum fünf Meter
		\item (W) Kommunikation in gesamtem Raum möglich
	\end{itemize}
\item Energieversorgung
	\begin{itemize}
		\item (F) Kabellose Stromversorgung
		\item (W) Akkus im Gerät aufladbar
	\end{itemize}
\item Geometrie
	\begin{itemize}
		\item (F) Handgerät, zweihändige Bedienung
	\end{itemize}
\item Budget
	\begin{itemize}
		\item (F) Leistungsaufnahme: Gerätelaufzeit mindestens eine Stunde Dauerbetrieb
	\end{itemize}
	\begin{itemize}
		\item (F) Gewicht weniger als 500 Gramm
	\end{itemize}
	\begin{itemize}
		\item (F) Kosten maximal \EUR{50} exklusive STM32F4 Discovery Board
	\end{itemize}
	\begin{itemize}
		\item (F) Entwicklungs- und Herstellungszeit: Auslieferung spätestens 12. Semesterwoche im SS 2015
	\end{itemize}
\item Herstellungsprozess
	\begin{itemize}
		\item Elektronik in Eigenarbeit herstellen, Software selbst implementieren
	\end{itemize}
\item Standards
	\begin{itemize}
		\item (F) Einhaltung von in Deutschland gültigen, für den Betrieb notwendigen
	 Standards
	\end{itemize}
\item Sicherheit
	\begin{itemize}
		\item (F) keine gesundheitliche Gefährdung von Personen durch das Gerät
	\end{itemize}
\item Ergonomie
	\begin{itemize}
		\item (W) einfache Bedienbarkeit
		\item (W) ansprechendes Aussehen
		\item (W) ergonomische Handhabung
	\end{itemize}
\end{itemize}