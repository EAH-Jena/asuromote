\chapter{Treffen am 13.10.2014}
\section{Anwesende Personen}
\begin{itemize}
	\item Rene Winkler
	\item Mathias Fugmann
	\item Junlong Yin
\end{itemize}

\section{Besprochen bzw. Bearbeitet}
\begin{itemize}
	\item Design-Schritt 2 begonnen
	\item Erstellung der Konzeptskizzen für drei Varianten
\end{itemize}

\section{Ausarbeitung von drei Konzepten}
\subsection{Konzept Blau}
Zu sehen in \autoref{fig:blau_v1}.
\begin{itemize}
	\item soll einfache Bedienung ermöglichen
	\item Benutzung eines 4-Wege-Analogjoysticks zur Handsteuerung
	\begin{itemize}
		\item Beschaffbarkeit eventuell ein Problem!
	\end{itemize}
	\item ein günstiges LCD könnte Verwendung finden und alle interessanten 	Informationen anzeigen
	\item hoher Aufwand an Bluetooth-Modulen (3 Stück) - Kostenfaktor!
\end{itemize}

\subsection{Konzept Grün}
Zu sehen in \autoref{fig:gruen_v1}.
\begin{itemize}
	\item benutzt günstigeres Funkmodul zur Asuro-Kommunikation
	\item Verbindung zum PC kann entweder am Asuro oder an der Fernsteuerung erfolgen - flexibel
	\begin{itemize}
		\item Akku beider Geräte kann direkt per USB geladen werden - großer Vorteil!
	\end{itemize}
	\item Konzept kann optional die Neigungssensoren zur Steuerung benutzen, weil D-Pad eher eingeschränkt
	\begin{itemize}
		\item neue Darstellungsmöglichkeiten auf Display - könnte dann ein größeres Dot-Matrix-Display sein
	\end{itemize}
\end{itemize}

\subsection{Konzept Rot}
Zu sehen in \autoref{fig:rot_v1}.
\begin{itemize}
	\item Design verwendet zwei BT-Module, sowie die USB-Verbindung
	\begin{itemize}
		\item gutes User-Handling
	\end{itemize}
	\item Neigungssensoren sind einzige Steuerungsmöglichkeit
	\begin{itemize}
		\item man könnte Buttons evtl. durch ein Touchscreen ersetzen
		\item keine beweglichen Teile (günstig!)
		\item Touchscreen evtl. teurer
	\end{itemize}
	\item Herstellung für nicht-standard-Gehäuseform evtl. 3D-Drucker möglich!
\end{itemize}
