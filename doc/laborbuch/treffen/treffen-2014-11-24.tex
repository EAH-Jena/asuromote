\chapter{Treffen am 24.11.2014}
\section{Anwesende Personen}
\begin{itemize}
	\item Stefan Biereigel
	\item Rene Winkler
	\item Mathias Fugmann
	\item Junlong Yin
\end{itemize}

\section{Besprochen bzw. Bearbeitet}
Rene hat die Ansteuerung der SPI-Einheit und des Beschleunigungssensors nun in Treiber- und Hardwareebene getrennt, sodass eine autarke Weiterverwendbarkeit des SPI-Codes unabhängig vom Beschleunigungssensor gewährleistet ist.

Die Schnittstelle auf SPI-Ebene beläuft sich auf Initialisierungs- und Lese/Schreib-Routinen, welche der Treiber für den Beschleunigungssensor verwendet. Dieser stellt selbst nach oben hin Funktionen zum Auslesen der einzelnen Kanäle, zur Initialisierung sowie zur Sensorinitialisierung zur Verfügung.
